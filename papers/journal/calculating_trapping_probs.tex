
Both fully- and semi-autonomous operation of optical tweezers require
the capability to trap a particle and plan its path through the
workspace; avoiding collisions with other components in real-time.
The probability that a given particle will be trapped within a
predictable spatial region about the laser beam focus, or {\em
  trapping probability}, is a critical aspect of this real-time motion
planning problem.  As it is infeasible to determine trapping
probabilities through real-world experiments due to a large parameter
space (including, for example, initial particle position, neighboring
workspace component existence, motion of laser), we have developed a
computational framework in which trapping probabilities can be
determined through optical tweezers simulation.

\subsection{Simulated Particle Motion}
Our initial code operates on glass and silica microspheres, as these are
well-studied in the literature~\cite{wright1994parametric}.  Furthermore,
properties of these spheres can be accurately modeled using theoretically- and
experimentally-verified equations.

In general, any moving component in the workspace will experience hydrodynamic
forces coupled with a rapidly fluctuating force, the result of frequent and
numerous collisions with surrounding liquid molecules.  We can model these
closely connected forces using Langevin's equation~\cite{gardiner1985handbook,
langevin1908brownianmotion}. We use model given in ~\cite{balijepalli2010stochastic}.
Given a velocity $V$, we describe the change in
velocity over time as:
\begin{equation}
\label{eq:langevins-equation}
 \frac{dV(t)}{dt} = -\frac{\gamma}{m}V(t) + \frac{\xi}{m}\Gamma(t)
\end{equation}

The hydrodynamic forces over time ($t$), are a function of velocity
($V$), mass ($m$), and drag ($\gamma$).  The drag coefficient $\gamma$
is given by Stokes' Law~\cite{girault1979finite} as $\gamma = 6 \pi
\eta r$, with $\eta$ representing the fluid viscosity as a function of
temperature and $r$ representing the radius of the silica sphere, in
this case.  The stochastic, rapidly fluctuating force $\Gamma$ is
scaled by a constant $\xi$ satisfying the fluctuation-dissipation
theorem~\cite{weissbluth1989photon}, which is defined as $\xi =
\sqrt{2 \gamma K_B T}$, with $K_B$ representing Boltzmann's
constant~\cite{grassia2001dissipation}.  The stochastic term $\Gamma$
prevents a direct analytic solution to Langevin's equation; as such,
our framework uses a finite difference expression of
Equation~\ref{eq:langevins-equation}.  
The stochastic term $\Gamma$ is substituted by a appropriately scaled normal distribution
$N(0,1/\delta t)$ where $\delta t$ is the finite time step. The term  
$1/\sqrt{\delta t}$ is absorbed into a scaling constant as described in~\cite{balijepalli2010stochastic}.


%The hydrodynamic forces over time ($t$), are a function of velocity
%($V$), mass ($m$), and drag ($\gamma$).  The drag coefficient $\gamma$
%is given by Stokes' Law~\cite{girault1979finite} as $\gamma = 6 \pi
%\eta r$, with $\eta$ representing the fluid viscosity as a function of
%temperature and $r$ representing the radius of the silica sphere, in
%this case.  The stochastic, rapidly fluctuating force $\Gamma$ is
%scaled by a constant $\xi$ satisfying the fluctuation-dissipation
%theorem~\cite{weissbluth1989photon}, which is defined as $\xi =
%\sqrt{2 \gamma K_B T}$, with $K_B$ representing Boltzmann's
%constant~\cite{grassia2001dissipation}.  The stochastic term $\Gamma$
%prevents a direct analytic solution to Langevin's equation; as such,
%our framework uses a finite difference expression of
%Equation~\ref{eq:langevins-equation}.  We substitute a normal
%distribution $N(0,1)$ for $\Gamma$ and alter the scaling constant
%$\xi$ appropriately to include $\delta t$, the finite time step.

We combine the finite difference form of Langevin's equations with an
enumeration of all other forces affecting a silica sphere in the workspace to
yield Equation~\ref{eq:numerical-integration-equation}.  The external force
factor $F_{ext}$ includes the constant (for a sphere) forces of gravity and
buoyancy, as well as the all-important optical trapping force 
applied by the laser to the particle.

%\begin{equation}
%\label{eq:numerical-integration-equation}
% \frac{V(t + \delta t) - V(t)}{\delta t}  = -\frac{\gamma}{m}V(t) + \frac{1}{m}
%\sqrt{\frac{2 \gamma K_B T}{\delta t}} N(0,1) + \frac{F_{ext}}{m} 
%\end{equation}

\begin{equation}
\label{eq:numerical-integration-equation}
 A(t + \delta t)  = -\frac{\gamma}{m}V(t) + \frac{1}{m}
\sqrt{\frac{2 \gamma K_B T}{\delta t}} N(0,1) + \frac{F_{ext}}{m} 
\end{equation}


We explicitly integrate Equation~\ref{eq:numerical-integration-equation}
to obtain a simulation of particle movement over time.  In practice, the most
popular explicit integration techniques include fourth-order Runge-Kutta
methods~\cite{jameson1981numerical}, the Gear predictor-corrector
method~\cite{allen1990computer}, and second-order ``velocity'' Verlet
integration~\cite{verlet1968computer}.  Each is appropriate in certain
situations; as such, we use Verlet integration
(Equations~\ref{eq:velocity-verlet-integration-position}
and~\ref{eq:velocity-verlet-integration-velocity}) due to its conservation of
energy at larger time steps and ease of computation, as discussed
in~\cite{balijepalli2010stochastic}.  Since the time step has a direct effect on
the number of iterations required to run a simulation (and thus its runtime), we
are interested in maximizing time step size while maintaining tight error
bounds.

\begin{equation}
 \label{eq:velocity-verlet-integration-position}
 X(t + \delta t) = X(t) + V(t)\delta t + \frac{1}{2} A(t) \delta t^2 + O(\delta
t^4)
\end{equation}

\begin{equation}
 \label{eq:velocity-verlet-integration-velocity}
 V(t + \delta t) = V(t) + \frac{A(t + \delta t) + A(t)}{2} \delta t + O(\delta
t^2)
\end{equation}

Given velocity $V(t)$ and external force factor $F_{ext}$, we compute the acceleration
$A(t + \delta t )$ at the next time step using
Equation~\ref{eq:numerical-integration-equation}. From here, the
velocity Verlet method provides the next position, $X(t + \delta t)$,
and velocity, $V(t + \delta t)$, in a single pass.


The characteristic time scale of our entire model is given by the
relaxation time $\frac{m}{\gamma}$, with $\gamma = 6 \pi \eta r$ the drag
coefficient of Stokes' equations; this is the time needed for a particle's
initial velocity to converge to thermal equilibrium.  In our experiments, the
numerical integration time step $\delta t$ is set to the nearest multiple of
$100$ ns such that $\delta t << \frac{m}{\gamma}$. Choosing such a small $\delta
t$ provides an opportunity to observe interesting nonequilibrium behavior in the
simulation.  Furthermore, this small time step decreases maximum error in both
velocity and position, bounded by the velocity Verlet to $O(\delta t^2)$ and
$O(\delta t^4)$, respectively.

%\paragraph{Calculation of Force Fields.}
Our particle simulation framework accepts either a continuous or discretely
sampled force field.  In our experiments, we choose to represent the optical
field discretely.  Force values are provided by
numerically integrating the basic scattering and gradient forces of
radiation (see Ashkin's seminal paper~\cite{ashkin1986observation}).  Such
integration is
typically done by tracing representative rays of light from the laser
position through the workspace to the laser focus -- and possibly
through microspheres in the workspace.  By tracing the reflection and
refraction of these simulated rays, a reliable estimate of both the
scattering and gradient forces can be computed.  In our prototype
implementation, we rely on code written
in~\cite{banerjee2009generating} to obtain these forces.

\subsection{Trapping Probability Estimates}

To estimate trapping probabilities, particle trajectory simulation
is performed multiple times at a given point in the parameter
space. The simulation is conducted by starting the particle at the
designated location. The trapping probability is estimated as a ratio
of the number the times the sphere gets trapped by the laser beam over
the total number of trials.

Trapping is a complex phenomenon due to the Brownian motion of the
particles. If a beam is held stationary for an indefinite period of
time, then particles far away from the beam are likely to wander
into the beam due to Brownian motion and eventually be
trapped. Particles continue to exhibit Brownian motion even after they
have been trapped. Therefore, trapped particles eventually jump out of
the trap at practically useful laser powers.  Many different notions
of trapping probability can be defined based on the context. For the
purpose of this paper, we are mainly interested in a notion of
trapping probability which is relevant from the point of view of path
planning. In this application, the laser beam continues to
move. Hence, the time available for trapping is relatively small. For
the purpose of this paper, we have done all trapping probability
estimates for fixed finite period of time. Methodology presented in
this paper can be easily used to compute trapping probability as a
function of the available trapping time.

We have assumed that the Brownian motion inside the trap is
negligible. Hence, the mean time to escape of the trap is quite large
with respect to the planning horizon. Therefore, we do not account for
the possibility of a trapped particle escaping the trap in our
calculations.  The simulation infrastructure has a built-in capability
to simulate Brownian motion under the force field. Therefore, it can
automatically account for the situations where a particle will escape
the trap if the simulation is performed over the long period of time.
For the figures produced in this paper, we assume a particle to be
trapped if the probability of a particle escaping the laser within
$\SI{1}{\deci\second}$ is less than $\frac{1}{N}$, where $N$ is the
number of simulated trajectories per grid position.



%\documentclass{article}
%\begin{document}
%\todo[inline]{How we calculate trapping probabilities and why it's important}

% Both fully- and semi-autonomous operation of optical tweezers require the
% capability to trap a particle and plan its path through the workspace, avoiding
% collisions with other components, in real-time.  The probability that a given
% particle will be trapped within a predictable spatial region about the laser
% beam focus, or {\em trapping probability}, is a critical aspect of this
% real-time motion planning problem.  As it is infeasible to determine trapping
% probabilities through real-world experiments due to a large parameter space
% (including, for example, initial particle position, neighboring workspace
% component existence, motion of laser), we have developed a computational
% framework in which trapping probabilities can be determined through optical
% tweezers simulation.

% \subsection{Simulated Particle Motion}
% Our initial code operates on glass and silica microspheres, as these are
% well-studied in the literature~\cite{wright1994parametric}.  Furthermore,
% properties of these spheres can be accurately modeled using theoretically- and
% experimentally-verified equations.

% In general, any moving component in the workspace will experience hydrodynamic
% forces coupled with a rapidly fluctuating force, the result of frequent and
% numerous collisions with surrounding liquid molecules.  We can model these
% closely connected forces using Langevin's equation~\cite{gardiner1985handbook,
% langevin1908brownianmotion}. Given a velocity $V$, we describe the change in
% velocity over time as:
% \begin{equation}
% \label{eq:langevins-equation}
%  \frac{dV(t)}{dt} = -\frac{\gamma}{m}V(t) + \frac{\xi}{m}\Gamma(t)
% \end{equation}

% The hydrodynamic forces are a function of velocity, mass, and drag.  The drag
% coefficient $\gamma$ is given by Stokes' Law~\cite{girault1979finite} as $\gamma
% = 6 \pi \eta r$, with $\eta$ representing the fluid viscosity as a function of
% temperature and $r$ representing the radius of the silica sphere, in this case. 
% The stochastic, rapidly fluctuating force $\Gamma$ is scaled by a constant $\xi$
% satisfying the fluctuation-dissipation theorem~\cite{weissbluth1989photon},
% which is defined as $\xi = \sqrt{2 \gamma K_B T}$, with $K_B$ representing
% Boltzmann's constant~\cite{grassia2001dissipation}.  The stochastic term
% $\Gamma$ prevents a direct analytic solution to Langevin's equation; as such,
% our framework uses a finite difference expression of
% Equation~\ref{eq:langevins-equation}.  We substitute a normal distribution
% $N(0,1)$ for $\Gamma$ and alter the scaling constant $\xi$ appropriately to
% include $\delta t$, the finite time step.

% We combine the finite difference form of Langevin's equations with an
% enumeration of all other forces affecting a silica sphere in the workspace to
% yield Equation~\ref{eq:numerical-integration-equation}.  The external force
% factor $F_{ext}$ includes the constant (for a sphere) forces of gravity and
% buoyancy, as well as the all-important optical trapping force 
% applied by the laser to the particle.

% \begin{equation}
% \label{eq:numerical-integration-equation}
%  \frac{V(t + \delta t) - V(t)}{\delta t}  = -\frac{\gamma}{m}V(t) + \frac{1}{m}
% \sqrt{\frac{2 \gamma K_B T}{\delta t}} N(0,1) + \frac{F_{ext}}{m} 
% \end{equation}

% We explicitly integrate Equation~\ref{eq:numerical-integration-equation}
% to obtain a simulation of particle movement over time.  In practice, the most
% popular explicit integration techniques include fourth-order Runge-Kutta
% methods~\cite{jameson1981numerical}, the Gear predictor-corrector
% method~\cite{allen1990computer}, and second-order ``velocity'' Verlet
% integration~\cite{verlet1968computer}.  Each is appropriate in certain
% situations; as such, we use Verlet integration
% (Equations~\ref{eq:velocity-verlet-integration-position}
% and~\ref{eq:velocity-verlet-integration-velocity}) due to its conservation of
% energy at larger time steps and ease of computation, as discussed
% in~\cite{balijepalli2010stochastic}.  Since the time step has a direct effect on
% the number of iterations required to run a simulation (and thus its runtime), we
% are interested in maximizing time step size while maintaining tight error
% bounds.

% \begin{equation}
%  \label{eq:velocity-verlet-integration-position}
%  X(t + \delta t) = X(t) + V(t)\delta t + \frac{1}{2} A(t) \delta t^2 + O(\delta
% t^4)
% \end{equation}

% \begin{equation}
%  \label{eq:velocity-verlet-integration-velocity}
%  V(t + \delta t) = V(t) + \frac{A(t + \delta t) + A(t)}{2} \delta t + O(\delta
% t^2)
% \end{equation}

% Given position $X(t)$ and velocity $V(t)$, we compute the
% acceleration $A(t + \delta t )$ at the next time step using
% Equation~\ref{eq:numerical-integration-equation}.  From here, the velocity
% Verlet method provides position $X(t + \delta t)$ and $V(t + \delta t)$ in
% a single pass.

% The characteristic time scale of our entire model is given by the
% relaxation time $\frac{m}{\gamma}$, with $\gamma = 6 \pi \eta r$ the drag
% coefficient of Stokes' equations; this is the time needed for a particle's
% initial velocity to converge to thermal equilibrium.  In our experiments, the
% numerical integration time step $\delta t$ is set to the nearest multiple of
% $100$ ns such that $\delta t << \frac{m}{\gamma}$. Choosing such a small $\delta
% t$ provides opportunity to observe interesting nonequilibrium behavior in the
% simulation.  Furthermore, this small time step decreases maximum error in both
% velocity and position, bounded by the velocity Verlet to $O(\delta t^2)$ and
% $O(\delta t^4)$, respectively.

% \paragraph{Calculation of Optical Trapping Forces.}
% Numerical integration of the basic scattering and gradient forces of radiation (see Ashkin's seminal
% paper~\cite{ashkin1986observation}) yields our desired optical trapping forces.  Such integration is typically done by
% tracing representative rays of light from the laser position through the workspace to the laser focus -- and possibly
% through microspheres in the workspace.  By tracing the reflection and refraction of these simulated rays, a
% reliable estimate of both the scattering and gradient forces can be computed.  In our prototype implementation, we rely
% on code written in~\cite{banerjee2009real} to obtain these forces.

% \subsection{Trapping Probability Estimates}
% S.K. writes about motivation here?

%\end{document}
