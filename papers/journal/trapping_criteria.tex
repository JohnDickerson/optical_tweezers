%\documentclass{article}
%\begin{document}

Reliably classifying a specific particle as ``trapped'' or ``not trapped'' is
necessary to provide accurate overall trapping probabilities.  To accomplish
this goal, we provide a simple parallelizable method that returns, once per
particle, a $0$ (not trapped) or $1$ (trapped) upon termination.

Similar to~\cite{banerjee2009generating}, the following conditions sufficiently
describe the termination criteria for our method:
\begin{enumerate}
 \item \label{termination-fallout} Particle falls to the bottom of the
workspace $\rightarrow 0$.
 \item \label{termination-geometric} Particle strays outside the bounds
of the force field $\rightarrow 0$.
 \item \label{termination-timeout} Total experimental time is
expended without trapping $\rightarrow 0$.
 \item \label{termination-trapped} Particle is trapped $\rightarrow 1$.
\end{enumerate}

These termination criteria are supressed in certain
situations. Specifically, criterion~\ref{termination-fallout} is
ignored if the particle remains affected by the laser forces.
Furthermore, criterion~\ref{termination-geometric} is supressed if
either the particle is sufficiently close to the trap focus (as it
will be affected by strong optical gradient forces) or if the laser
has nontrivial horizontal velocity (as the cone of the laser might
move to intersect with the particle).

Criterion~\ref{termination-trapped} returns when, informally, the
optical forces imparted on a particle by the laser effectively
overpower all other acting forces (\emph{e.g.}, Brownian motion,
gravity, and buoyancy).  For a stationary laser (velocity $v =
\textbf{0}$), this amounts to a trapped particle remaining within
fixed distance $d^v_z$ beneath the laser and distance $d^v_{xy}$ along
the X- and Y-axes.  For a laser with nontrivial horizontal or vertical
velocity, both $d^v_z$ and $d^v_{xy}$ change considerably as a
function of both laser velocity and particle size.

The calculation of these trapping boundaries will potentially change
from experiment to experiment due to, for example, an adjustment in
the power of the laser; however, we assume that such environmental
variables will remain static across a single experimental run.  We do
expect both the horizontal and vertical components of the laser's
velocity to change over the course of a single experiment. As such, we
adopt a strategy of one-time computations of radial and axial trapping
bounds $d^v_{xy}$ and $d^v_{z}$ for different base laser velocities,
relying on interpolation to provide bounds for any velocity.

For a stationary laser, we place $1024$ particles at the origin and
execute a version of our full parallel simulation, described in
Section~\ref{sec:execution-main}, for $50$ms.  Over this time period,
all of these particles fall into the optical trap; our simulation
records the variation in both radial and axial displacement of the
particle.  Upon termination, we select the maximum of the maximum
radial and axial displacements as our final, stationary trapping
bounds.  This approach generalizes to a moving laser through
supression of criterion~\ref{termination-geometric}, as defined above.

%\end{document}
