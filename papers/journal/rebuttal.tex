%%% use twocolumn and 10pt options with the asme2ej format
\documentclass[10pt]{article}

\usepackage{amsmath}
\usepackage{amsfonts}
\usepackage{graphicx}
\usepackage{siunitx}

%\makeatletter
%\newif\if@restonecol
%\makeatother
%\let\algorithm\relax
%\let\endalgorithm\relax
%\usepackage{algorithm2e}

\usepackage[colorlinks]{hyperref}
%\usepackage[colorinlistoftodos, textwidth=4cm, shadow]{todonotes}

\title{Response to reviewer comments for ``Speeding Up Particle
  Trajectoy Simulations under Moving Force Fields using GPUs''}
\begin{document}
\maketitle

\section*{}
We wish to thank the reviewers for their very insightful comments.

\subsection*{Reviewer 1 Concerns:}
\subsubsection*{General comments:}

While reviewer 1 brings up several valid points concerning the optimization of our CPU
based implementation, the goal of this work, from a computational
standpoint, was to demonstrate that the problem of estimating trapping
probabilities under the given physical model, can benefit from GPU-based parallel computation.
To this end, we concentrate more on exploiting
parallelism where it exists, and examining the scaling behavior
(i.e. number of threads vs. running time) of the simulation.  The
authors believe that it is this scaling behavior, more than the
absolute performance numbers, which shows the promise of performing
these simulations in a massively parallel environment; regardless of 
the underlying hardware.


\subsubsection*{Specific Comments:}
\begin{enumerate}

\item The CPU code was reasonably optimized for cache locality.  We
  made the effort, where possible, to keep memory access local and
  contiguous.

\item No attempt was made to run the problem using OpenCL on both
  architectures.  The authors who wrote the code for the GPU
  simulation ran all of the experiments on Nvidia hardware, and, in
  conjunction with their previous experience, this made CUDA a natural
  choice.  Also, when this manuscript was originally written and
  submitted, the CPU-based OpenCL SDK was in a somewhat preliminary
  and unoptimized state; it may have actually detracted from the 
  performance of the CPU-based simulator.
  
\item The CPU implementation was not parallelized.  Rather, we wanted
  to draw an explicit comparison between the performance of a
  single-threaded vs. highly parallelized version of the simulator.
  This allowed us to avoid issues concerning which CPU-based
  parallelization framework to use, and the implications of the
  performance overhead they impose.  We have updated the manuscript to
  make the single-threaded nature of the CPU based implementation more
  clear.

\item Neither the CPU nor GPU code was implemented in a matrix/vector
  formulation.  It is absolutely true that formulating our simulation
  in terms linear algebraic operations and implementing these using a
  high performance library could significantly increase the
  performance of the CPU based simulator.  However, we would also like
  to point out that previous work suggests that similar performance
  gains can be realized on the GPU by formulating one's problems in
  terms of linear algebraic operations.

\end{enumerate}



\subsection*{Reviewer 2 Concerns:}

\begin{enumerate}

\item Our original CPU results were computed using
  single-precision floats and mathematical operations.  Thus, the
  timing comparisons are between the single-precision CPU and
  single-precision GPU simulators. Additionally, we've included an
  analysis of single vs. double precision simulation results to
  address the concern that a significant loss of accuracy may occur.

\item The CPU implementation against which we compared is
  single-threaded.  Please see point (3) above for an explanation of
  this comparison.  Also, we've updated the manuscript to make this
  point clear.

\item We've added new figures to address the accuracy concerns, and
  have cleaned up the original figures.

\end{enumerate}



\subsection*{Reviewer 3 Concerns:}
\begin{itemize}

\item Defined variables in section 3.1

\item 
	%\textcolor{red}{Halfway down the 2nd column of p. 3 you say
    %given $X(t)$, compute using Equation 2, but equation 2 doesn't use
    %$X(t)$ as an input.}	
	We have clarified the Equation 2 and the statement mentioned.


\item Clarified nature of how ``trapped'' particles are handled in our simulation

\item 
	%\textcolor{red}{A the end of p.3, 1st column, you talk about scaling to include $\delta_{t}$, 
    %but equation 2 also takes a square root? This should be explained.}
	We have added details regarding the equation 2 issue.

\item 
	%\textcolor{red}{Why was $0.25$ chose as the interval (end of p.4, column 2)?}
 The diameter of each nano particle is $\SI{5}{\micro\meter}$. We choose the grid interval to be $\SI{0.25}{\micro\meter}$
  because it is relatively smaller than the size of each particle and will give us more accurate result.


\item The purpose of figure 1 is to explain, at a high level, the GPU
  architecture, and how this hardware design affects algorithms and
  implementation decisions.  Such diagrams are generally useful to
  those who are unfamiliar with GPU architecture.

\item 
	%\textcolor{red}{Portions of section 4.4 imply that there can be multiple lasers, but I don't 
    %think the algorithm works for multiple lasers.}
Currently, our work is only focused on a single laser. 

\item 
	%\textcolor{red}{p.6, column 2 halfway down, what is $N_{s}$ Did
    %you plot the results to confirm the expected Gaussian
    %distribution?  What is the maximum error in?  Doesn't it depend on
    %the mean value?}
$N_{s}$ has been replaced by N (which is the number of particles used).
 We have also plotted the result and verified that the result is a Gaussian distribution.


\item ``each of these particles falls into the optical trap'' $\to$ ``all of these particles fall into the optical trap''

\item Equation 5 classifies the set of trapped particles using
  standard set notation.  It explains that the set of trapped
  particles (TRAPPED) is the set containing all particles $p_{i}$,
  such that particle $p_{i}$ was trapped by laser $l$, which is moving
  with velocity $v_{l}$.

\item As stated in section 4.4, $1024$ simulations are sufficient to
  obtain a $95\%$ confidence interval (i.e. $95\%$ of the true trapping 
  probabilities reside within a deviation of $\pm 0.03125$ from our trapping
  probability estimate).

\item A detailed comparison of the CPU and GPU computations has been
  made, and both a visual relative error plot, as well as a numerical
  analysis of the relative numerical differences between the two has
  been presented. 

\item The issue of precision and the axis ranges have been addressed.

\item We changed the wording of the section which describes the
  GPU-based implementation's speed up to reflect the fact that we are
  suggesting, rather than asserting, what may account for the
  superlinear speed increase we observed.

\item The flatness between $\sim 100$ and $\sim 1000$ in figure $10$ can be attributed to the fact
  that until $\sim 1000$, the GPU does not reach full resource utilization.  Thus, between $\sim 100$ and 
  $\sim 1000$, we can add more particles essentially ``for free'' without incurring any additional resource
  contention.

\item 
	%\textcolor{red}{Do the results get more accurate with more
    %particles?  How do you know how much?}
The nano particles are affected by the Brownian motion which follows the normal distribution. 
So as the number of particles increase the accuracy of the simulation is expected increase.

\item Perhaps the sparsity of labels on the log-log scale make the
  timings a bit difficult to extract.  The x-axis runs from $1$ to
  $32768$.  The simulator takes $12.27$ seconds for $8$ particles and
  $23.0048$ seconds for $4096$ particles, so the runtime slows by a factor
  of $\sim 1.875$ between $8$ and $4096$ particles.

\end{itemize}

\end{document}
